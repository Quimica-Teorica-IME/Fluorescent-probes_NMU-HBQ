\documentclass{standalone}
\usepackage{xcolor}
\usepackage{tikz, pgfplots}
\usetikzlibrary{plotmarks, arrows}
\usetikzlibrary{calc, decorations, plotmarks, fit, positioning, shapes.geometric}
\pgfplotsset{compat=1.4}

% ===========================================================================
% colour definitions

\definecolor{BL}{HTML}{0099e6}
\definecolor{B}{HTML}{006699}
\definecolor{BD}{HTML}{004466}
\definecolor{BR}{HTML}{0080FF}  % like RoyalBlue

\definecolor{RL}{HTML}{d65454}
\definecolor{R}{HTML}{893636}
\definecolor{RD}{HTML}{ad1737}  % like Maroon

\definecolor{GL}{HTML}{10d05d}
\definecolor{G}{HTML}{009933}
\definecolor{GD}{HTML}{004d1a}

\definecolor{Y}{HTML}{D0D030}
\definecolor{O}{HTML}{FFA500}
\definecolor{T}{HTML}{4DC0FF}
\definecolor{OL}{HTML}{FFAA80}
\definecolor{P}{HTML}{DD88BB}
\definecolor{V}{HTML}{703676}

% ===========================================================================

% settings for the coordinate system
\pgfplotsset{
every linear axis/.append style={
width=7.000 cm, height=3.0cm,xmin=0, xmax=6,
scale only axis, axis on top,
  yticklabel style={
    /pgf/number format/fixed,
    /pgf/number format/zerofill,
    /pgf/number format/precision=1},
  legend style={at={(1.01,0.00)}, anchor=south west, draw=none, fill=black!5, inner sep=0.5pt, outer sep=0.5pt},
  legend columns=1, legend cell align=left},
}
\pgfplotsset{colormap={CI}{color=(white); color=(blue!50); color=(O!70);}}

% ===========================================================================

\tikzstyle{orb} = [ellipse, draw, color=black, fill=black!5, inner sep=-1pt, align=center]
\tikzstyle{arrow} = [-open triangle 45, black, thick]
\newcommand{\incMO}[1]{\includegraphics[width=4.5cm, trim=1cm 1cm 1cm 1cm, clip=true]{#1}}

\begin{document}
\begin{tikzpicture}
\begin{axis}[
      anchor=south west, at={(0,0)},
      height=3.0cm,
      xlabel={}, ylabel={Energy (eV)},
      xtick={-10}, xticklabels={},
      %colorbar % uncomment for colorbar as legend
      ]
\addplot[mark=-, only marks, mark options={scale=2, line width=2pt}]  table[x index=0, y index=2] {Om_bar_data.txt};

% Uncomment to plot oscillator strengths as shading
%\addplot+[solid, mesh, point meta=explicit, no markers, line width=5cm, shader=flat corner] table[x expr=\thisrowno{0}-0.5, y expr=2, meta expr=\thisrowno{3}] {Om_bar_data.txt};
\end{axis}
\begin{axis}[
      anchor=north west, at={(0,0)},
      xlabel={State}, ylabel={Character},
      ytick={0.0,0.2,...,0.8},
      xtick pos=left, xtick align=outside,
xtick={1, 2, 5},
xticklabels={$1(1)A$, $2(1)A$, $5(1)A$},
ymin=-0.0, ymax=1,
ybar stacked, bar width=0.933 cm]
\addplot[ybar, draw=black, fill=blue] table[x index=0, y index=4] {Om_bar_data.txt};
\addlegendentry{CTAB};
\addplot[ybar, draw=black, fill=pink] table[x index=0, y index=5] {Om_bar_data.txt};
\addlegendentry{CTBA};
\addplot[ybar, draw=black, fill=green] table[x index=0, y index=6] {Om_bar_data.txt};
\addlegendentry{LEA};
\addplot[ybar, draw=black, fill=orange] table[x index=0, y index=7] {Om_bar_data.txt};
\addlegendentry{LEB};

% Uncomment for double excitations
% \addplot[ybar, draw=black, fill=yellow!50] table[x index=0, y index=8] {Om_bar_data.txt};
% \addlegendentry{2-el.};
% Handles for arrows
\node at (axis cs: 1,-0.15) (h1) {};
\node at (axis cs: 2,-0.15) (h2) {};
\node at (axis cs: 5,-0.15) (h3) {};
\end{axis}



% Template for drawing orbital pictures
%    You can use the \incMO command defined above
% \node[orb, anchor=north west, at={(-1cm,-4.8cm)}] (S1e) {incMO};
% \node[orb, below of=S1e, anchor=north, node distance=2cm] (S1h) {incMO};
% \node[orb, right of=S1e, anchor=west, xshift=0.7cm, yshift=-0.3cm] (S2e) {incMO};
% \node[orb, below of=S2e, anchor=north, node distance=2cm] (S2h) {incMO};
% \node[orb, right of=S2e, anchor=west, xshift=0.7cm, yshift=0.3cm] (S3e) {incMO};
% \node[orb, below of=S3e, anchor=north, node distance=2cm] (S3h) {incMO};
%

% \draw (S1e) to (S1h);
% \draw (S2e) to (S2h);
% \draw (S3e) to (S3h);

% \draw[arrow] (S1e) to [out=90, in=-90] (h1);
% \draw[arrow] (S2e) to [out=90, in=-90] (h2);
% \draw[arrow] (S3e) to [out=90, in=-90] (h3);
\end{tikzpicture}
\end{document}
